% \iffalse meta-comment
%<*internal>
\iffalse
%</internal>
%<*readme>
This package is based on Robin Fairbairns footmisc package

It allow to personnalize footnote on LaTeX
%</readme>
%<*internal>
\fi
\def\nameofplainTeX{plain}
\ifx\fmtname\nameofplainTeX\else
  \expandafter\begingroup
\fi
%</internal>
%<*install>
\input l3docstrip.tex
\keepsilent
\askforoverwritefalse
\preamble
---------------------------------------------------------------
The footmisx package
---------------------------------------------------------------
\endpreamble
\postamble
Copyright
\endpostamble
\generate{
  \file{\jobname.sty}{\from{\jobname.dtx}{package}}
}
%</install>
%<install>\endbatchfile
%<*internal>
\generate{
  \file{\jobname.ins}{\from{\jobname.dtx}{install}}
}
\nopreamble\nopostamble
\generate{
  \file{README.md}{\from{\jobname.dtx}{readme}}
}
\ifx\fmtname\nameofplainTeX
  \expandafter\endbatchfile
\else
  \expandafter\endgroup
\fi
%</internal>
%<*driver|package>
\RequirePackage{expl3}[2015/09/11]
\RequirePackage{xparse}
\RequirePackage{filehook}[2011/01/09]
\ProvidesExplPackage{footmisx}{2016/12/01}{20161201}{miscellany of footnote facility}
\ExplSyntaxOff
%</driver|package>
%    \end{macrocode}
%
% Code to enable LaTeX processing of the file without the intervention
% of a driver file.
%    \begin{macrocode}
%<*driver>
\setcounter{errorcontextlines}{999}
\documentclass{l3doc}
\usepackage{hyperref}
%    \end{macrocode}
%
% we limit the things that we will index
%    \begin{macrocode}
\DoNotIndex{\#,\@MM,\@cclv,\@gobble,\@ifnextchar,\@ifundefined}
\DoNotIndex{\|,\advance,\begingroup,\bgroup,\box,\csname}
\DoNotIndex{\dagger,\ddagger,\def,\divide,\dp}
\DoNotIndex{\edef,\egroup,\ifx,\else,\fi,\endcsname,\endgroup,\end}
\DoNotIndex{\endinput,\ensuremath,\expandafter}
\DoNotIndex{\gdef,\global,\hbox,\hskip,\hss,\ht}
\DoNotIndex{\ifcase,\or,\ifdim,\ifhbox,\ifhmode,\ifnum,\ifvbox,
            \fi        ,\fi   ,\fi    ,\fi     ,\fi   ,\fi    }
\DoNotIndex{\ifvoid,\fi,\kern,\let,\long,\loop}
\DoNotIndex{\MessageBreak,\newbox,\newcommand,\newcounter}
\DoNotIndex{\newdimen,\newif,\newskip,\newtoks,\noexpand}
\DoNotIndex{\P,\p@,\par,\penalty,\protect,\providecommand}
\DoNotIndex{\relax,\renewcommand,\S,\setbox,\setcounter}
\DoNotIndex{\skip,\space,\the,\typeout,\vbox,\vskip}
\DoNotIndex{\wd,\xdef,\@}
\GetFileInfo{footmisx.dtx}
\EnableCrossrefs
% to get documented source of the package, comment out the next line,
% and uncomment the following one; otherwise, create yourself
% (somewhere on your LaTeX input path) a file ltxdoc.cfg that contains
% simply          \AtBeginDocument{\AlsoImplementation}
%\OnlyDescription
\AlsoImplementation
\setcounter{StandardModuleDepth}{1}
\begin{document}
\DocInput{footmisx.dtx}
\end{document}
%</driver>
%    \end{macrocode}
% \fi
%
%
%
% \title{\texttt{footmisx} ---\\
%        a portmanteau package\\
%        for customising footnotes in \LaTeX\thanks{This file has
%          version number \fileversion, last revised \filedate}}
% \author{Bastien Roucari\`es formerly
%     Robin Fairbairns\thanks{University of Cambridge Computer
%     Laboratory, William Gates Building, J.\,J. Thompson Avenue,
%     Cambridge, CB2 0FD, UK
%     (\texttt{rf10<snail-shape>cam.ac.uk})}}
% \maketitle
%
% \tableofcontents{}
%
% \section*{Copyright statement}
%
% \noindent Program: \texttt{footmisx.dtx}\par
% \noindent Copyright 1995 1996 1998 1999 2001--2003 2008 2009 Robin Fairbairns
% \noindent Copyright 2016 Bastien Roucari\`es
%
% This program is offered under the terms
% of the LaTeX Project Public License, version 1.3c of this license or
% (at your option) any later version.  The latest version of this
% license is in http://www.latex-project.org/lppl.txt, and version
% 1.3c or later is part of all distributions of LaTeX version
% 2005/12/01 or later.
%
% This work has the LPPL maintenance status `author-maintained'.
%
% \section*{History}
%
% This package is based of  Robin Fairbairns footmisc package.
%
% This package originated as support of a personal project, which Robin
% was switching to \LaTeX{} 2e over the Christmas holiday period of
% 1993, using the first \ensuremath{\beta} release.
%
% In its first form, it was known as the ``footnote'' package, but by
% the time I had released it to CTAN, that name had already been
% used by a package written by Mark Wooding.  So the package is now
% known (as you can see) as ``footmisc''.
%
% In 2016 it was forked as footmisx
%
% \section{User interface~--- package options}
%
% The \textsf{footmisx} package provides several different
% customisations of the way foonotes are represented in \LaTeXe{}
% documents (the sources of the code in this package are various, but
% all of it has been massaged by the author; where the code comes from
% elsewhere, there are attributions given below, somewhere or other).
%
% The interface to the
% package's options is mostly rather simple~--- each one is presented as an
% option in the |\usepackage| command, and for most, nothing else
% needs to be done.  For example, to use a useful
% and consistent set, the author invokes the package with the
% command |\usepackage[perpage,para,symbol*]{footmisx}|.
%
% For a small number of options, there are additional parameters
% available; these are described in the subsections below.
%
% \subsection{Option \texttt{perpage}}
% \label{sec:perpagedoc}
%
% This option resets footnote numbering for each page of the document.
% It needs at least two passes to do this correctly (though it comes
% as close as possible on the first pass).  You generally have to make
% two passes with \LaTeX{} anyway, to get the cross-references right,
% so an additional pass for this purpose shouldn't cause any
% additional problem.  The option includes code to report that
% `\emph{Label(s) may have changed}', which will help the poor user to
% realise that (yet) another run is in order.
%
% \subsection{Option \texttt{para}}
%
% This option (derived from code by Dominik Wujastyk and Chris Rowley)
% causes footnotes to be typeset as a single paragraph at the bottom
% of the page on which they occur.  In the case that there is only one
% footnote on the page, no effect will be observed.  However, if there
% are several footnotes on the page, they will be run together in the
% page foot, each introduced by its footnote mark.  The original
% demand for the option came from the needs of those preparing
% critical editions; such documents typically have large numbers of
% small footnotes, which look ridiculous if each is typeset in a
% paragraph of its own; in most other disciplines, such multiplicities
% of footnotes represent mere self-indulgence: the author of this
% package is disgracefully guilty of this.
%
% Please note that ``old'' \LaTeX{} installations may have problems
% with the algorithm for \texttt{para} footnotes on very wide pages
% (for example, those used by the \textsf{a0poster} class).  Recent
% \LaTeX{} installations use an improved technique that is believed
% not to be susceptible to this problem.
%
% \subsection{Option \texttt{side}}
%
% This option (suggested by Frank Mittelbach) causes footnotes to be
% typeset using the \cs{marginpar} command: this has the advantage
% that the note appears close to its ``call-up'', but has all the
% disadvantages associated with the \cs{marginpar} command (which
% consumes `float' slots, and doesn't always place itself correctly at
% the top of pages in two-sided documents).  Since the measure in
% which the footnote is to be typeset is likely to be pretty narrow,
% users of the \texttt{side} option are recommended also to use the
% \texttt{ragged} option, to avoid ugly spacing and line breaks.
%
% There is a further problem (apart from the occasional failure to
% place the marginal note on the correct side of the page) in
% two-sided documents: one would like `raggedness' to appear
% differently in different margins (setting the left, rather than the
% right, side ragged in the left margin).  (The author would welcome
% suggestions on means of addressing the problem.)
%
% \subsection{Option \texttt{ragged} and \cs{footnotelayout}}
%
% The package provides facilities for ragged right setting of
% footnotes (so long as the \texttt{para} option isn't in effect).
% The change is effected by use of the command \cs{footnotelayout};
% the package inserts this command into the start of the argument of
% \cs{footnotetext} (in effect: \cs{footnote} works, roughly, by
% calling the guts of \cs{footnotetext} at its end).
%
% If you want to use some special effect other than ragged right, feel
% free to change \cs{footnotelayout} yourself: some intriguing (and
% completely undesirable) results are no doubt available.  Change the
% setting simply by use of
% \cs{renewcommand}\cs{footnotelayout}\texttt{\dots}\@.  The
% \texttt{ragged} option simply sets \cs{footnotelayout} to set
% \cs{raggedright} or \cs{RaggedRight} as appropriate.  (If you intend
% to use the \textsf{ragged2e} package, load it before
% \textsf{footmisx}~--- if \textsf{footmisx} finds \cs{RaggedRight} is
% available, it automatically uses it in place of \cs{raggedright}.)
%
% \subsection{Option \texttt{symbol}}
%
% This option simply establishes that footnotes are `labelled' by
% a symbol sequence.  The command used is equivalent to that
% suggested in \LaTeX{} manuals such as Lamport's (the job performed
% by the option is very simple, and doesn't really need a package).
%
% Using symbols to `number' your footnotes can be problematic: there
% is a limited number of symbols, and \LaTeX{} will report an error if
% your footnotes exceed that limit.  To avoid such problems, consider
% the \texttt{symbol*} option, or the \cs{setfnsymbol} command (see
% the next two sections), or number your footnotes by the page (see
% section~\ref{sec:perpagedoc}).
%
% \subsection{Option \texttt{symbol*}}
% \label{doc-symbol*}
%
% This is the \texttt{symbol} option, but with protection against the
% tedium that arises because of the instability of the
% \texttt{perpage} option.  When executing the \texttt{perpage}
% option, the package often allocates footnotes to the wrong pages,
% only to correct itself on a later run (having warned the user of the
% need for the later run with a `\emph{Label(s) may have changed}'
% message).  In these circumstances the \texttt{symbol} option is
% prone to producing \LaTeX{} errors, which stop processing, and
% confound automatic generation procedures.  In the same situation,
% the \texttt{symbol*} option produces information messages and a
% warning message at end document, and the user may scan the log for
% those messages \emph{after} processing has stabilised.  The option
% produces numbers (17 and higher, in the case of the default symbol
% set) in place of symbols, when the footnote number is too large.
%
% \subsection{The \cs{setfnsymbol} and \cs{DefineFNsymbols} commands}
% \label{footnote-symbols}
%
% These commands permit the definition and use of alternative
% (ordered) sets of symbols for numbering footnotes.  \LaTeX{} of
% course comes with such a set ready-defined, but the choice of
% symbols isn't universally loved.
%
% You may define a set of symbols with the \cs{DefineFNsymbols}
% command.  \LaTeX{}'s default set would be defined by the command:
% \begin{center}
%   \verb|\DefineFNsymbols*{lamport}|%
%   \unskip\verb|{*\dagger\ddagger\S\P|\texttt{\char`\\\char`\|\%}\\
%   \unskip\verb|          {**}{\dagger\dagger}{\ddagger\ddagger}}|
% \end{center}
% Defined this way, the symbol set produces a ``counter too large''
% error; a robust version of the set (cf.~the \texttt{symbol*} option
% (see \ref{doc-symbol*}) using the \cs{DefineFNsymbols} command
% without the optional |*|.
% You may select a set of symbols by use of the \cs{setfnsymbol}
% command; so to restore use of the default set, you would type:
% \begin{center}
%   \verb|\setfnsymbol{lamport}|
% \end{center}
%
% This package defines a small selection of alternative sets of
% symbols, using \cs{DefineFNsymbol}:
% \begin{center}
%   \begin{tabular}{ll}
%     \texttt{bringhurst} & $*\,\dagger\,\ddagger\,\S\,\|\,\P$ \\
%     \texttt{chicago}    & $*\,\dagger\,\ddagger\,\S\,\|\,\#$ \\
%     \texttt{wiley}      & $*\,\mathop{**}\,\dagger\,\ddagger\,\S\,\P\,\|$
%   \end{tabular}
% \end{center}
% together with a version of Lamport's original set that, with doubled
% versions of $\S$ and $\P$, and tripled versions of everything but
% the vertical bars, provides a symbol range to cover counters up to
% 16.
%
% This last set, known as \texttt{lamport*} is selected as the default
% symbol set by the package.
%
% \subsection{Option \texttt{bottom}}
%
% This option forces footnotes to the bottom of the page; this is only
% noticeably useful in case that \cs{raggedbottom} is in effect, when
% \LaTeX{} would normally set the footnotes a mere
% \cs{skip}\cs{footins} distant from the bottom of the text.
%
% There's a further infelicity in \LaTeX{}'s placing of footnotes of
% the bottom of pages: if a bottom float appears on a page, \LaTeX{}
% places the footnote \emph{above} it.  The \texttt{bottom} option
% places the footnote at the foot of the page.
%
% \subsection{Option \texttt{marginal}}
%
% This option adjusts the position of footnote mark relative to the
% start of the line in which they appear (the the option is
% incompatible with option \texttt{para}, for obvious reasons).
%
% When this option is in effect, the footnote is set
% \cs{footnotemargin} relative to the left margin of the page; the
% default setting for \cs{footnotemargin} is -0.8em, which means that
% the footnote mark will be set jutting 0.8em into the margin.  If
% \cs{footnotemargin} is a positive length, the footnote mark will be
% set with its right edge \cs{footnotemargin} from the margin.  (In
% the absence of the option, \cs{footnotemargin} is set to 1.8em; you
% may change that value with a \cs{setlength} command.)
%
% \subsection{Option \texttt{flushmargin}}
%
% This option is as option marginal, but sets the footnote marker
% flush with, but just inside the margin from, the text of the
% footnote.
%
% \subsection{Option \texttt{hang}}
%
% This option sets the footnote mark flush with the margin, and makes
% the body of the footnote hang at an indentation of
% \cs{footnotemargin} (if that is a positive distance), or the width
% of the marker (if \cs{footnotemargin}$\leq0$).  The option code
% itself leaves \cs{footnotemargin} at its default value of 1.8em.
%
% The footnote itself may of course be longer than one paragraph; if
% so, the paragraphs will be separated by the vertical space specified
% by \cs{hangfootparskip}, and the second and subsequent paragraphs
% are indented by \cs{hangfootparindent}.  Default values are:
% \begin{center}
% \begin{tabular}{ll}
%   \cs{hangfootparskip}   & 0.5\cs{baselineskip} \\
%   \cs{hangfootparindent} & 0em
% \end{tabular}
% \end{center}
% The user may redefine these values (using
% \cs{renewcommand}): it is best to use the font-size-dependent
% measures (multiples of \cs{baselineskip} for the skip, multiples of
% |em| for the indent).  Note that the default has only one of the two
% values non-zero; both zero may result in easily-missed paragraph
% breaks, and both non-zero is not generally thought to be a
% good-looking option.
%
% \subsection{Option \texttt{norule}}
%
% This option suppresses the `normal' footnote rule, and advances
% \cs{skip}\cs{footins} a bit to compensate
%
% \subsection{Option \texttt{splitrule}}
%
% This option makes puts a full-width rule above the split-off part of
% a split footnote.  (Remember that split footnotes don't happen if
% you're doing paragraph footnotes.)
%
% The option provides three different \cs{footnoterule} commands:
% \begin{center}
% \begin{tabular}{ll}
% \cs{mpfootnoterule} & for use in minipages \\
% \cs{pagefootnoterule} & for normal footnotes on regular pages \\
% \cs{splitfootnoterule} & for the tail of a split footnote
% \end{tabular}
% \end{center}
% By default, \cs{mpfootnoterule} and \cs{pagefootnoterule} retain the
% original definition of \cs{footnoterule} (which nay have been
% modified by a \texttt{norule} option), while \cs{splitfootnoterule}
% becomes a full-width rule.
%
% \subsection{The \texttt{stable} option}
%
% This option deals with the problem of placing footnotes in section
% titles (and so on).  While there is (sometimes, just) justification
% for putting footnotes in titles, \LaTeX's treatment of the content
% of titles militates against them.  Of course, the title argument is
% ordinarily a moving one, and \cs{footnote} is a fragile command, but
% the real problem comes from the way the argument actually moves~---
% which is to two places.  The argument moves to the table of
% contents, where the footnote will (at least) look odd.  But the
% argument also moves to the marks that make up page headers, etc.,
% and \emph{there} it creates havoc, since page headers are executed
% in page make-up, and page make-up \emph{must not} create footnotes.
%
% If you use the \texttt{stable} option, the footnote won't move to
% the table of contents or the page headers, but it will be typeset
% correctly within the title itself.
%
% The situation with \cs{footnotemark} is less dire (it could in
% principle appear in page headers, for example); footnote marks
% appearing on pages other than where their text appears are none the
% less confusing, and the stable option treats \cs{footnotemark} in
% the same way that it treats \cs{footnote}.
%
% \subsection{The \texttt{multiple} option}
%
% This option deals with the case where the author needs to type
% things like
% \begin{verbatim}
%   mumble\footnote{blah}\footnote{grumble}
% \end{verbatim}
% Without special treatment, \LaTeX{} would output something like
% \begin{quote}
%   mumble\textsuperscript{1314}
% \end{quote}
% \noindent What the \texttt{multiple} option makes of the above is
% \begin{quote}
%   mumble\textsuperscript{13,14}
% \end{quote}
% which is what most people would expect.  The comma separator
% actually derives from the definition of \cs{multfootsep}, which
% may be changed by \cs{renewcommand} if the option is in effect.
%
% The option also treats \cs{footnotemark} in the same way.
%
% \subsection{User interface~--- miscellaneous commands}
%
% The package also defines some miscellaneous footnote-related
% commands.  The present group provides alternative means of producing
% footnote marks: \cs{footref} and \cs{mpfootnotemark}.
%
% When you're in a minipage, \cs{footnote} numbers run according to the
%  minipage's own footnote counter, and the marks are set in italic
% letters.  However, the numbers used by \cs{footnotemark} make
% reference to the `main'
% footnote counter, and are set in whatever is the current style for
% that: this behaviour often surprises, and there's no obvious way in
% standard \LaTeX{} to ``get around'' it.  The command
% \cs{mpfootnotemark} gets around this problem in a minipage, by
% generating footnote marks in the same way as those used by
% \cs{footnote}.
%
% In fact, making reference to footnotes in
% general can be problematic: it can be done by noting down the
% value of the footnote
% marker in a counter (or the like) and then using the value in a
% subsequent \cs{footnotemark} or \cs{mpfootnotemark}.  This is a
% tedious way of going about things, and doesn't allow representation
% of all possible forms of footnote mark; \cs{footref} is a form of
% reference command that sets the reference as if it were a footnote.
% The label should be set \emph{within} the argument of the footnote
% command that is being labelled:
% \begin{verbatim}
% ...\footnote{Note text\label{fnlabel}}
% ...
% ... potato head\footref{fnlabel}
% ... potato head\footref*{fnlabel} without link.
% \end{verbatim}
%
% \section{User interface~--- interactions with other packages}
%
% The \textsf{footmisx} package modifies several parts of the \LaTeX{}
% kernel; what gets modified depends on the options you select.  This
% behaviour can cause problems with other packages, particularly those
% that also modify the kernel.
%
% Known interactions are:
% \begin{description}
% \item[\normalfont\textsf{setspace}] The \textsf{setspace} package
%   modifies the way line spacing is calculated in footnotes.
%   \textsf{footmisx} knows about this, and preserves the change.
%   
%   Note that \textsf{setspace} could be loaded anywhere except between 
%   \textsf{hyperref} and \textsf{footmisx}.
% \item[\normalfont\textsf{memoir} class] The class emulates
%   \textsf{setspace}, and we detect that emulation and deal with it
%   in the same way as \textsf{setspace}.
% \item[\normalfont\textsf{hyperref}] The \textsf{hyperref} package
%   has ambitions to make hyperlinks from footnote marks to the
%   corresponding footnote body;  \textsf{footmisx} knows about this,
%   and preserves hyperlink.
% \item[\normalfont\textsf{manyfoot}] The \textsf{manyfoot} package
%   permits several independent sequences of footnotes.  Some
%   preliminary work towards interworking with \textsf{footmisx} has
%   been completed, but more remains to be done at the time of
%   writing.
% \end{description}
%
% Know incompatibilities are:
% \begin{description}
% \item[\normalfont\textsf{footnpag}] Replace with option \texttt{perpage}. You could also
% use package \textsf{perpage} and \tn{MakePerPage{footnote}} command.
% \item[\normalfont\textsf{pagefoots}] Replace with option \texttt{perpage}. You could also
% use package \textsf{perpage} and \tn{MakePerPage{footnote}} command.
% \item[\normalfont\textsf{pfnote}] Replace with option \texttt{perpage}. You could also
% use package \textsf{perpage} and \tn{MakePerPage{footnote}} command.
% \item[\normalfont\textsf{fnpara}] Replace with option \texttt{para}. 
% \item[\normalfont\textsf{fnpos}] Replace with option \texttt{bottom}.
% \end{description}
% \StopEventually{}
%
% \section{Code: Preliminaries}
%
% Well~--- here we go: let's make the package file:
%
%    \begin{macrocode}
%<*driver|package>
%    \end{macrocode}
%
% Enable lt3docstrip stripping
%    \begin{macrocode}
%<@@=footmisx>
%    \end{macrocode}
%
% \subsection{Error messages}
%
% Here we define some error messages:
%    \begin{macrocode}
\ExplSyntaxOn
\msg_new:nnn {footmisx} {not expected definition}
   {The~macro~#1~is~not~defined~as~expected.\\
    The~macro~#1~could~not~be~modified~by~footmisx.\\
    Please~report~a~bug. \\
    Get:\\
    #2\\
    Expect:\\
    #3\\
   }
\msg_new:nnn {footmisx} {too old}
   {The~package~#1~is~too~old~and~thus~not~supported.\\
    Please~install~a~newer~version~(at~least~version~#2)}
\msg_new:nnn {footmisx} {incompatible}
   {The~package~#1~is~incompatible~with~footmisx.\\
    Use~the~following~workarround~#2}
%    \end{macrocode}
%
% \subsection{Check package loaded and package version}
%
% \subsubsection{Helpers}
%
% Protect against too old package by checking version if package is loaded.
% \begin{function}{\@@_error_if_too_old:NN}
%   \begin{syntax}
%     \cs{@@_error_if_too_old:NN} \Arg{package} \Arg{version}
%   \end{syntax}
% Check minimal version and raise an error in case of too old.
%    \begin{macrocode}
\cs_new_nopar:Nn \@@_error_if_too_old:NN {
  \AtBeginDocument{
    \ExplSyntaxOn
    \@ifpackageloaded{#1}{
      \@ifpackagelater{#1}{#2}{}
        {\msg_error:nnnn{footmisx}{too old}{#1}{#2}}
    }{}
    \ExplSyntaxOff
  }
}
%    \end{macrocode}
% \end{function}
% Protect against incompatible package.
% \begin{function}{\@@_error_if_incompatible:NN}
%   \begin{syntax}
%     \cs{@@_error_if_incompatible:NN} \Arg{package} \Arg{version}
%   \end{syntax}
% Check minimal version and raise an error in case of too old.
%    \begin{macrocode}
\cs_new_nopar:Nn \@@_error_if_incompatible:NN {
  \@ifpackageloaded{#1}{
    \msg_error:nnnn{footmisx}{incompatible}{#1}{#2}
  }{}
  \AtBeginOfPackageFile{#1}{
    \msg_error:nnnn{footmisx}{incompatible}{#1}{#2}
  }
  \AtBeginDocument{
    \ExplSyntaxOn
    \@ifpackageloaded{#1}{
      \msg_error:nnnn{footmisx}{incompatible}{#1}{#2}
    }{}
    \ExplSyntaxOff
  }
}
%    \end{macrocode}
% \end{function}
%
% \subsubsection{Checks}
%
% Check now some minimal supported version
%    \begin{macrocode}
\@@_error_if_too_old:NN{hyperref}{2016/06/24}
\@@_error_if_too_old:NN{setspace}{2011/12/19}
\@@_error_if_incompatible:NN{footnpag}{Use footmisx with option perpage}
\@@_error_if_incompatible:NN{pagefoots}{Use footmisx with option perpage}
\@@_error_if_incompatible:NN{pfnotes}{Use footmisx with option perpage}
\@@_error_if_incompatible:NN{fnpara}{Use footmisx with option para}
\@@_error_if_incompatible:NN{fnpos}{Use footmisx with option bottom}
%    \end{macrocode}
%
% \subsection{Check definition}
%
% \subsubsection{Helper}
%
% We check definition of some overwriten function. Protect against too recent package.
% \begin{function}{\@@_error_if_neq:NNc}
%   \begin{syntax}
%     \cs{@@_error_if_neq:NNc} \Arg{sequence_1} \Arg{sequence_2} \marg{csname}
%   \end{syntax}
%   Compare \Arg{sequence_1} and \Arg{sequence_2}. Call \marg{csname} if not
%   equal in \cs{cs_if_eq:NNTF} sense.
%
%   \marg{csname} should be a \cs{foo:nnxxx} function like \cs[module=l3basics]{msg_critical:nnxxx}
%    \begin{macrocode}
\cs_new:Npn \@@_msg_if_neq:NNc #1#2#3 {
  \cs_if_eq:NNTF {#1} {#2} {}
                 {\use:c{#3}
                   {footmisx}
                   {not expected definition}
                   {\exp_not:N #1}
                   {\exp_not:N {\cs_meaning:N #1}}
                   {\exp_not:N {\cs_meaning:N #2}}
                 }
}
%    \end{macrocode}
% \end{function}
% \begin{function}{\@@_fatal_if_neq:NN}
%   \begin{syntax}
%     \cs{@@_fatal_if_neq:NN} \Arg{sequence_1} \Arg{sequence_2}
%   \end{syntax}
%   Compare \Arg{sequence_1} and \Arg{sequence_2}. Call \cs[module=l3basics]{msg_fatal:nnxxx} if not
%   equal in \cs{cs_if_eq:NNTF} sense.
%    \begin{macrocode}
\cs_new:Nn \@@_fatal_if_neq:NN {
  \@@_msg_if_neq:NNc {#1} {#2} {msg_fatal:nnxxx}
}
%    \end{macrocode}
% \end{function}
% \begin{function}{\@@_critical_if_neq:NN}
%   \begin{syntax}
%     \cs{@@_critical_if_neq:NN} \Arg{sequence_1} \Arg{sequence_2}
%   \end{syntax}
%   Compare \Arg{sequence_1} and \Arg{sequence_2}. Call \cs[module=l3basics]{msg_critical:nnxxx} if not
%   equal in \cs{cs_if_eq:NNTF} sense.
%    \begin{macrocode}
\cs_new:Nn \@@_critical_if_neq:NN {
  \@@_msg_if_neq:NNc {#1} {#2} {msg_critical:nnxxx}
}
%    \end{macrocode}
% \end{function}
% \begin{function}{\@@_error_if_neq:NN}
%   \begin{syntax}
%     \cs{@@_error_if_neq:NN} \Arg{sequence_1} \Arg{sequence_2}
%   \end{syntax}
%   Compare \Arg{sequence_1} and \Arg{sequence_2}. Call \cs[module=l3basics]{msg_error:nnxxx} if not
%   equal in \cs{cs_if_eq:NNTF} sense.
%    \begin{macrocode}
\cs_new:Nn \@@_error_if_neq:NN {
  \@@_msg_if_neq:NNc {#1} {#2} {msg_error:nnxxx}
}
%    \end{macrocode}
% \end{function}
% \begin{function}{\@@_warning_if_neq:NN}
%   \begin{syntax}
%     \cs{@@_warning_if_neq:NN} \Arg{sequence_1} \Arg{sequence_2}
%   \end{syntax}
%   Compare \Arg{sequence_1} and \Arg{sequence_2}. Call \cs[module=l3basics]{msg_warning:nnxxx} if not
%   equal in \cs{cs_if_eq:NNTF} sense.
%    \begin{macrocode}
\cs_new:Nn \@@_warning_if_neq:NN {
  \@@_msg_if_neq:NNc {#1} {#2} {msg_warning:nnxxx}
}
%    \end{macrocode}
% \end{function}
% \begin{function}{\@@_note_if_neq:NN}
%   \begin{syntax}
%     \cs{@@_note_if_neq:NN} \Arg{sequence_1} \Arg{sequence_2}
%   \end{syntax}
%   Compare \Arg{sequence_1} and \Arg{sequence_2}. Call \cs[module=l3basics]{msg_note:nnxxx} if not
%   equal in \cs{cs_if_eq:NNTF} sense.
%    \begin{macrocode}
\cs_new:Nn \@@_note_if_neq:NN {
  \@@_msg_if_neq:NNc {#1} {#2} {msg_note:nnxxx}
}
%    \end{macrocode}
% \end{function}
% \begin{function}{\@@_log_if_neq:NN}
%   \begin{syntax}
%     \cs{@@_log_if_neq:NN} \Arg{sequence_1} \Arg{sequence_2}
%   \end{syntax}
%   Compare \Arg{sequence_1} and \Arg{sequence_2}. Call \cs[module=l3basics]{msg_log:nnxxx} if not
%   equal in \cs{cs_if_eq:NNTF} sense.
%    \begin{macrocode}
\cs_new:Nn \@@_log_if_neq:NN {
  \@@_msg_if_neq:NNc {#1} {#2} {msg_log:nnxxx}
}
%    \end{macrocode}
% \end{function}
%
% \subsubsection{Checks}
%
% Now we check if \tn{@makefnmark} has been modified. Do it between group in
% order to not pollute namespace.
%
%     \begin{macrocode}
\group_begin:
  \def\@@_copy_@makefnmark{\hbox{\@textsuperscript{\normalfont\@thefnmark}}}
  \@@_warning_if_neq:NN {\@makefnmark} {\@@_copy_@makefnmark}
\group_end:
%    \end{macrocode}
%
% Now we check if \tn{@footnotemark} has been modified. Do it between group in
% order to not pollute namespace.
%    \begin{macrocode}
\group_begin:
\@ifpackageloaded{hyperref}
%%hyperef case
{
  \def\@@_copy_@footnotemark{%
    \leavevmode
    \ifhmode\edef\@x@sf{\the\spacefactor}\nobreak\fi
    \stepcounter{Hfootnote}%
    \global\let\Hy@saved@currentHref\@currentHref
    \hyper@makecurrent{Hfootnote}%
    \global\let\Hy@footnote@currentHref\@currentHref
    \global\let\@currentHref\Hy@saved@currentHref
    \hyper@linkstart{link}{\Hy@footnote@currentHref}%
    \@makefnmark
    \hyper@linkend
    \ifhmode\spacefactor\@x@sf\fi
    \relax
  }%
}
%% latex case
{
  \def\@@_copy_@footnotemark{
    \leavevmode
    \ifhmode\edef\@x@sf{\the\spacefactor}\nobreak\fi
    \@makefnmark
    \ifhmode\spacefactor\@x@sf\fi
    \relax
  }
}
\@@_error_if_neq:NN {\@footnotemark} {\@@_copy_@footnotemark}
\group_end:
%    \end{macrocode}
% Now we check if \tn{footnotemark} has been modified. Do it between group in
% order to not pollute namespace.
%    \begin{macrocode}
\group_begin:
\def\@@_copy_footnotemark{\@ifnextchar [%]
    \@xfootnotemark
    {%
      \stepcounter{footnote}%
      \protected@xdef\@thefnmark{\thefootnote}%
      \@footnotemark
    }%
}
\@@_error_if_neq:NN {\footnotemark} {\@@_copy_footnotemark}
\group_end:
%    \end{macrocode}
%
% \subsection{Load hook now and at end of some package}
%
% \begin{function}{\@@_error_if_neq:NN}
%   \begin{syntax}
%     \cs{@@_AtEndOfPackageFile_and_now:NN} \Arg{package} \Arg{code}
%   \end{syntax}
%   Run \Arg{code} now and at end of package \Arg{package}. 
%     \begin{macrocode}
\cs_new:Npn \@@_AtEndOfPackageFile_and_now:NN #1#2 {
#2
\AtEndOfPackageFile{#1}{#2}
}
%    \end{macrocode}
% \end{function}

% \subsection{Generalities}
%
% We save a private macro using csname in order to use lt3doc
%    \begin{macrocode}
\cs_set_eq:cc {@@_par} {@ @ par}
\ExplSyntaxOff
%    \end{macrocode}
% We need a token register in case we have to patch \cs{@makecol}:
%    \begin{macrocode}
\newtoks\FN@temptoken
%    \end{macrocode}
%
% \begin{macro}{\protected@writeaux}
% This command is defined for future compatibility with Matt Swift's
% \textsf{newclude} package (still, after all this time, not out of
% beta status).
%    \begin{macrocode}
\providecommand\protected@writeaux{%
  \protected@write\@auxout
}%    \end{macrocode}
% \end{macro}
%
% \begin{macro}{\footnotemargin}
% Finally, we define the length used by the \texttt{marginal} option,
% and initialise it as if we've not had the option.
%    \begin{macrocode}
\newdimen\footnotemargin
\footnotemargin1.8em\relax
%    \end{macrocode}
% \end{macro}
%
% \section{Package options}
%
% Most of the code of the package is contained within the option
% processing, one way or another (that which isn't, is executed after
% \cs{ProcessOptions} as a result of flags set in the option
% processing).
%
% \subsection{The \texttt{symbol} option}
%
% This is a declaration that appears in the original \LaTeX{} book.
% Since it appeared in the old |pagefoots.sty| (presumably since it
% goes so naturally with the |perpage| option), I've added this
% trivial piece of customisation to the package.
%
%    \begin{macrocode}
\DeclareOption{symbol}{\renewcommand\thefootnote{\fnsymbol{footnote}}}
%    \end{macrocode}
%
% \subsection{The \texttt{symbol*} option}
%
% The robust version of the \texttt{symbol} option: if the current
% `symbol' option doesn't provide enough variants, use arabic footnote
% number.  We use a robust version of the ``extended ordinary'' symbol set,
% described later (in section~\ref{footnote-symbols}).
%    \begin{macrocode}
\DeclareOption{symbol*}{%
  \renewcommand\thefootnote{\@fnsymbol\c@footnote}%
  \AtEndOfPackage{\setfnsymbol{lamport*-robust}}%
}
%    \end{macrocode}
%
% \subsection{The \texttt{para} option}
%
% The basis of the code for this option comes from \TeX{}book, p.398
% ff.~(``Dirty Tricks''), though it does (of course) avoid
% redefining |\\| which has some other (somewhat significant) uses in
% \LaTeX{}!  The user should be aware of
% Knuth's note on the limitations of this method of doing the job: the
% \TeX{} stack is used four times per footnote, and the stack is
% limited (see the \TeX{}book, p.300 ff.).  If you have very large
% numbers of footnotes (in the hundreds), and encounter the error
% ``|! TeX capacity exceeded, sorry (... save size ...)|'', you may
% need to break your text into smaller sections and compile the
% separately.  Fortunately (say the comments on the original
% |fnpara.sty|) this is very easy to do with \LaTeX{}, provided that
% you reset the footnote counter to make the joins seamless.
%
% \begin{macro}{\ifFN@para}
% Define the |para| option: now simply sets a marker for use later
% when defining the option's auxiliary code and when patching the
% output routine and so on.
%    \begin{macrocode}
\newif\ifFN@para  \FN@parafalse
\DeclareOption{para}{\ifFN@sidefn
    \PackageError{footmisx}{Option "\CurrentOption" incompatible with
      option "side"}%
      {I shall ignore "\CurrentOption"}%
  \else
    \FN@paratrue
  \fi
}
%    \end{macrocode}
% \end{macro}
%
% \subsection{The \texttt{side} option}
%
% \begin{macro}{\ifFN@sidefn}
% Simply changes the behaviour of \cs{@footnotetext}; incompatible
% with paragraph footnotes.
%    \begin{macrocode}
\newif\ifFN@sidefn  \FN@sidefnfalse
\DeclareOption{side}{\ifFN@para
    \PackageError{footmisx}{Option "\CurrentOption" incompatible with
      option "para"}%
      {I shall ignore "\CurrentOption"}%
  \else
    \FN@sidefntrue
  \fi
}
%    \end{macrocode}
% \end{macro}
%
% \subsection{The \texttt{ragged} option}
%
% \begin{macro}{\footnotelayout}
% A very simple option that merely changes the definition of one
% macro.  Note detection of the presence of the \textsf{ragged2e}
% package.
%    \begin{macrocode}
\let\footnotelayout\@empty
\DeclareOption{ragged}{%
  \@ifundefined{RaggedRight}%
    {\renewcommand\footnotelayout{\linepenalty50 \raggedright}}%
    {\renewcommand\footnotelayout{\linepenalty50 \RaggedRight}}%
}
%    \end{macrocode}
% \end{macro}
%
% \subsection{The \texttt{perpage} option}
%
% \begin{macro}{\ifFN@perpage}
% A footnote-numbering modification: a new algorithm replacing one
% from Brian T. Schellenberger, which has proved to be flawed.  We
% simply set a marker here, and define code later depending on the
% state of the marker (see section \ref{sec:perpage-code}).
%    \begin{macrocode}
\newif\ifFN@perpage
\FN@perpagefalse
\DeclareOption{perpage}{%
  \FN@perpagetrue
}
%    \end{macrocode}
% \end{macro}
%
% \subsection{The \texttt{bottom} option}
%
% \begin{macro}{\ifFN@bottom}
% All this needs to do is to set a flag to say that it should happen
%    \begin{macrocode}
\newif\ifFN@bottom  \FN@bottomfalse
\DeclareOption{bottom}{%
  \FN@bottomtrue
}
%    \end{macrocode}
% \end{macro}
%
% \subsection{The \texttt{marginal} option}
%
% Again, the processing of the option is pretty trivial:
%    \begin{macrocode}
\DeclareOption{marginal}{%
  \footnotemargin-0.8em\relax
}
%    \end{macrocode}
%
% \subsection{The \texttt{flushmargin} option}
%
% Again, the processing of the option is pretty trivial:
%    \begin{macrocode}
\DeclareOption{flushmargin}{%
  \footnotemargin0pt\relax
}
%    \end{macrocode}
%
% \subsection{The \texttt{hang} option}
%
% \begin{macro}{\ifFN@hangfoot}
% We need a switch, since \cs{@makefntext} needs to be patched.
%    \begin{macrocode}
\newif\ifFN@hangfoot  \FN@hangfootfalse
\DeclareOption{hang}{%
  \FN@hangfoottrue
}
%    \end{macrocode}
% \end{macro}
%
% \begin{macro}{\hangfootparskip}
% \begin{macro}{\hangfootparindent}
% Layout parameters for hanging footnotes; \cs{hangfootparskip} and
% \cs{hangfootparindent} are (respectively) values to use for
% \cs{parskip} and \cs{parindent} when in hanging footnotes.
%    \begin{macrocode}
\newcommand*\hangfootparskip{0.5\baselineskip}
\newcommand*\hangfootparindent{0em}%
%    \end{macrocode}
% \end{macro}
% \end{macro}
%
% \subsection{The \texttt{norule} option}
%
% Pretty simple too\dots
%
%    \begin{macrocode}
\DeclareOption{norule}{%
  \renewcommand\footnoterule{}%
  \advance\skip\footins 4\p@\@plus2\p@\relax
}
%    \end{macrocode}
%
% \subsection{The \texttt{splitrule} option}
%
% \begin{macro}{\split@prev}
% This is from a posting by Donald Arseneau dated 13 November 1996.
% The code relies on the fact that \LaTeX{} only uses inserts for
% footnotes, so that if any insert is going to be split, it's going to
% be a footnote.
%    \begin{macrocode}
\DeclareOption{splitrule}{%
  \gdef\split@prev{0}
%    \end{macrocode}
% \end{macro}
%
% \begin{macro}{\pagefootnoterule}
% \begin{macro}{\mpfootnoterule}
% \begin{macro}{\splitfootnoterule}
% Define defaults for the three footnote rules: note, we inherit the
% current state of \cs{footnoterule} for the two `regular' footnote
% defaults, and if we've been preceded by option \texttt{norule}, they
% will both become null\dots
%    \begin{macrocode}
  \let\pagefootnoterule\footnoterule
  \let\mpfootnoterule\footnoterule
  \def\splitfootnoterule{\kern-3\p@ \hrule \kern2.6\p@}
%    \end{macrocode}
% \end{macro}
% \end{macro}
% \end{macro}
%
% Now redefine \cs{footnoterule} to distinguish the three situations.
%    \begin{macrocode}
  \def\footnoterule{\relax
    \ifx \@listdepth\@mplistdepth
%    \end{macrocode}
%
% In a minipage
%    \begin{macrocode}
      \mpfootnoterule
    \else
      \ifnum\split@prev=\z@
%    \end{macrocode}
%
% Normal footnote on a regular page
%    \begin{macrocode}
        \pagefootnoterule
      \else
%    \end{macrocode}
%
% Second part of a split footnote
%    \begin{macrocode}
        \splitfootnoterule
      \fi
%    \end{macrocode}
%
% Remember a split for next page
%    \begin{macrocode}
      \xdef\split@prev{\the\insertpenalties}%
    \fi
  }%
}
%    \end{macrocode}
%
% \begin{macro}{\ifFN@stablefootnote}
% \subsection{The \texttt{stable} option}
%
% Simply set a flag: the code of this gets executed at the very end of
% the package.
%    \begin{macrocode}
\newif\ifFN@stablefootnote  \FN@stablefootnotefalse
\DeclareOption{stable}{\FN@stablefootnotetrue}
%    \end{macrocode}
% \end{macro}
%
% \subsection{The \texttt{multiple} option}
%
% \begin{macro}{\ifFN@multiplefootnote}
% Again, simply set a flag, for code that gets executed at the very
% very very end of the package.
%    \begin{macrocode}
\newif\ifFN@multiplefootnote  \FN@multiplefootnotefalse
\DeclareOption{multiple}{\FN@multiplefootnotetrue}
%    \end{macrocode}
% \end{macro}
%
% \subsection{The start of the endgame}
%
% Exercise the options that the user has requested\dots
%    \begin{macrocode}
\ProcessOptions
%    \end{macrocode}
%
% \section{Hacking kernel commands}
%
% Various standard commands (some of them internal ones) need to be
% hacked to achieve our effects, and we do all of this now, according
% to flags set in option processing.
%
% \subsection{The output routine}
%
% Now; do we need to mess about with the output routine?  If either
% |para| or |bottom| has been invoked, we do.
%    \begin{macrocode}
\let  \if@tempswa  \ifFN@bottom
\ifFN@para    \@tempswatrue \fi
\if@tempswa
%    \end{macrocode}
% \dots{} so we've patching to do.
%
% First, we ensure that \cs{@makecol} is as expected from the time at
% which these macros were written: since we're going to patch it, we
% had better be sure that we're patching the right thing.  (There was
% a minuscule change to the definition 1999, but this doesn't as far
% as I can tell make any difference to the semantics of the definition
% we base our patch on.)
%    \begin{macrocode}
  \@ifl@t@r\fmtversion{2005/12/01}{%
    \CheckCommand*\@makecol{\ifvoid \footins
        \setbox\@outputbox \box\@cclv
      \else
        \setbox\@outputbox \vbox{%
          \boxmaxdepth \@maxdepth
          \unvbox\@cclv
          \vskip \skip\footins
          \color@begingroup
            \normalcolor\footnoterule
            \unvbox\footins
          \color@endgroup
        }%
      \fi
      \let \@elt \relax
      \xdef\@freelist{\@freelist\@midlist}%
      \global\let\@midlist\@empty
      \@combinefloats
      \ifvbox\@kludgeins
        \@makespecialcolbox
      \else
        \setbox\@outputbox \vbox to\@colht{%
          \@texttop \dimen@\dp\@outputbox
          \unvbox\@outputbox
          \vskip -\dimen@\@textbottom
        }%
      \fi
      \global\maxdepth\@maxdepth
    }
  }{%
  \@ifl@t@r\fmtversion{2003/12/01}{%
    \CheckCommand*\@makecol{\ifvoid \footins
        \setbox\@outputbox \box\@cclv
      \else
        \setbox\@outputbox \vbox{%
          \boxmaxdepth\@maxdepth
          \@tempdima\dp\@cclv
          \unvbox\@cclv
          \vskip \skip\footins
          \color@begingroup
            \normalcolor
            \footnoterule
            \unvbox\footins
          \color@endgroup
        }%
      \fi
      \let \@elt \relax
      \xdef\@freelist{\@freelist\@midlist}%
      \global\let\@midlist\@empty
      \@combinefloats
      \ifvbox\@kludgeins
        \@makespecialcolbox
      \else
        \setbox\@outputbox \vbox to\@colht{%
          \@texttop
          \dimen@\dp\@outputbox
          \unvbox\@outputbox
          \vskip -\dimen@
          \@textbottom
        }%
      \fi
      \global\maxdepth\@maxdepth
    }%
  }{%
    \@ifl@t@r\fmtversion{1999/12/01}{%
      \CheckCommand*\@makecol{\ifvoid \footins
          \setbox\@outputbox \box\@cclv
        \else
          \setbox\@outputbox \vbox{%
            \boxmaxdepth\@maxdepth
            \@tempdima\dp\@cclv
            \unvbox\@cclv
            \vskip \skip\footins
            \color@begingroup
              \normalcolor\footnoterule
              \unvbox\footins
            \color@endgroup
          }%
        \fi
        \xdef\@freelist{\@freelist\@midlist}%
        \global\let\@midlist\@empty
        \@combinefloats
        \ifvbox\@kludgeins
          \@makespecialcolbox
        \else
          \setbox\@outputbox \vbox to\@colht{%
            \@texttop \dimen@\dp\@outputbox
            \unvbox\@outputbox
            \vskip -\dimen@\@textbottom
          }%
        \fi
        \global\maxdepth\@maxdepth
      }%
    }%
  }{%
      \CheckCommand*\@makecol{\ifvoid \footins
        \setbox\@outputbox \box\@cclv
      \else
        \setbox\@outputbox \vbox{%
          \boxmaxdepth\@maxdepth
          \unvbox\@cclv
          \vskip \skip\footins
          \color@begingroup
            \normalcolor\footnoterule
            \unvbox\footins
          \color@endgroup
        }%
      \fi
      \xdef\@freelist{\@freelist\@midlist}%
      \global\let\@midlist\@empty
      \@combinefloats
      \ifvbox\@kludgeins
        \@makespecialcolbox
      \else
        \setbox\@outputbox \vbox to\@colht{%
          \@texttop \dimen@\dp\@outputbox
          \unvbox\@outputbox
          \vskip -\dimen@\@textbottom
        }%
      \fi
      \global\maxdepth\@maxdepth
    }%
  }%
}
%    \end{macrocode}
%
% If we're doing paragraph footnotes, the output routine needs
% different code to place the actual text.  We prepare this code here,
% since it's potentially used in two different places.
%
% We prepare the code in a token register to be used at the
% appropriate place in the patching of \cs{@makecol}; thus it becomes
% a token register containing code to place stuff in a token register
%    \begin{macrocode}
  \ifFN@para
%    \end{macrocode}
%
% We make a box out of the paragraph of footnotes, and then stuff the
% contents of the box into that which is going to be \cs{ship}ped
% |out|.
%    \begin{macrocode}
    \FN@temptoken{%
      \toks@\expandafter{\the\toks@
        \vskip\skip\footins
        \color@begingroup
          \normalcolor\footnoterule
          \global\setbox\FN@tempboxc\vbox{\makefootnoteparagraph}%
          \unvbox\FN@tempboxc
        \color@endgroup
      }%
    }%
%    \end{macrocode}
%
% If we're not doing paragraph footnotes, we insert the little bit of
% code that would have been replaced by the stuff above:
%    \begin{macrocode}
  \else
    \FN@temptoken{%
      \toks@\expandafter{\the\toks@
        \vskip\skip\footins
        \color@begingroup
          \normalcolor\footnoterule
          \unvbox\footins
        \color@endgroup
      }%
    }%
  \fi
%    \end{macrocode}
%
% Now we start building up the revised version of \cs{@makecol}.  The
% definition starts out in \cs{toks@}; first the \textsf{bottom} version:
%    \begin{macrocode}
  \ifFN@bottom
    \toks@{\setbox\@outputbox \box\@cclv
      \xdef\@freelist{\@freelist\@midlist}%
      \global\let\@midlist\@empty
      \@combinefloats
      \ifvoid\footins
      \else
        \setbox\@outputbox \vbox\bgroup
          \boxmaxdepth\@maxdepth
          \unvbox\@outputbox
          \vfill\relax
    }
    \the\FN@temptoken
    \toks@\expandafter{\the\toks@\egroup\fi}
%    \end{macrocode}
%
% Not putting stuff at the bottom: footnotes are placed using the
% kernel's algorithm.
%    \begin{macrocode}
  \else
    \toks@{\ifvoid\footins
        \setbox\@outputbox\box\@cclv
      \else
        \setbox\@outputbox \vbox\bgroup
        \boxmaxdepth\@maxdepth
        \unvbox\@cclv
    }
    \the\FN@temptoken
%    \end{macrocode}
%
% Finally, close the \cs{setbox} and the \cs{ifvoid} and tag the parts
% of the definition  of \cs{@makecol} up to the end of the definition
% of the \textsf{bottom} version on to \cs{toks@}.
%    \begin{macrocode}
    \toks@\expandafter{\the\toks@
      \egroup
      \fi
      \xdef\@freelist{\@freelist\@midlist}%
      \global\let\@midlist\@empty
      \@combinefloats
    }%
  \fi
%    \end{macrocode}
%
% Finally, create the new definition from the resulting object with
% the remainder of the original \cs{@makecol} tagged on at the end.
%    \begin{macrocode}
  \toks@\expandafter{\the\toks@
    \ifvbox\@kludgeins
      \@makespecialcolbox
    \else
      \setbox\@outputbox \vbox to\@colht{%
        \@texttop \dimen@\dp\@outputbox
        \unvbox\@outputbox
        \vskip -\dimen@\@textbottom
      }%
    \fi
    \global\maxdepth\@maxdepth
  }
  \edef\@makecol{\the\toks@}
%    \end{macrocode}
%
% All of the above occurred conditionally on the `or' of
% \cs{ifFN@para} and \cs{ifFN@bottom}, so we now close the
% conditional.
%    \begin{macrocode}
\fi
%    \end{macrocode}
%
% \subsection{Support code for paragraph footnotes}
%
% This code used (most inefficiently) to be in the argument of the
% \cs{DeclareOption}; this no doubt comes of that code having been
% written over Christmas 1993\dots
%
% Now all executed under the |para| conditional set in the option
% declaration.
%    \begin{macrocode}
\ifFN@para
%    \end{macrocode}
%
% \begin{macro}{\FN@tempboxa}
% \begin{macro}{\FN@tempboxb}
% \begin{macro}{\FN@tempboxb}
% We need some temporary boxes, and \LaTeX{} only defines one
%    \begin{macrocode}
  \let\FN@tempboxa\@tempboxa
  \newbox\FN@tempboxb
  \newbox\FN@tempboxc
%    \end{macrocode}
% \end{macro}
% \end{macro}
% \end{macro}
%
% \begin{macro}{\footglue}
% A direct crib from the \TeX{}book:
%    \begin{macrocode}
  \newskip\footglue \footglue=1em plus.3em minus.3em
%    \end{macrocode}
% \end{macro}
%
% \begin{macro}{\@makefntext}
% The standard classes set the footnote mark flush with the text of
% the footnote, but that's not appropriate for paragraph footnotes, we
% find.
%
% There's not much point in patching this code from the original,
% since the only things it has in common with the original are the
% footnote mark and the footnote text (which last is the argument).
% Note that the \cs{leavevmode} isn't necessary except in the case of
% footnotes in minipages, which otherwise end up with the
% \cs{@makefnmark} being executed in restricted vertical mode, which
% results in its \cs{hbox} ending up in a line of its own.
%
%    \begin{macrocode}
  \long\def\@makefntext#1{\leavevmode
    \textfootmark{\@thefnmark}\nobreak
    \hskip.5em\relax#1%
  }
%    \end{macrocode}
% \end{macro}
%
%%%%%%%%%%%%%%%%%%%%%%%%%%%%%%%%%%%%%%%%%%%%%%%%%%%%%%%%%%%%%%%%%%%%%%%%%%%%%
%
% \begin{macro}{\footnotebaselineskip}
% We need to record a value for the baseline skip when in footnotes:
%    \begin{macrocode}
  \newdimen\footnotebaselineskip
  {%
    \footnotesize
    \global
      \footnotebaselineskip=\normalbaselineskip
  }
%    \end{macrocode}
% \end{macro}
%
% \begin{macro}{\makefootnoteparagraph}
% For use in the output routine
%    \begin{macrocode}
  \long\def\makefootnoteparagraph{\unvbox\footins \makehboxofhboxes
    \setbox\FN@tempboxa=\hbox{\unhbox\FN@tempboxa \removehboxes}
%    \end{macrocode}
% Now we are ready to set the paragraph:
%    \begin{macrocode}
    \hsize\columnwidth
    \@parboxrestore
    \baselineskip=\footnotebaselineskip
    \noindent
    \rule{\z@}{\footnotesep}%
    \unhbox\FN@tempboxa\par
  }
%    \end{macrocode}
% \end{macro}
%
% \begin{macro}{\makehboxofhboxes}
% \begin{macro}{\removehboxes}
% Support code for \cs{makefootnoteparagraph}
%    \begin{macrocode}
  \def\makehboxofhboxes{\setbox\FN@tempboxa=\hbox{}%
    \loop
      \setbox\FN@tempboxb=\lastbox
      \ifhbox\FN@tempboxb
      \setbox\FN@tempboxa=\hbox{\box\FN@tempboxb\unhbox\FN@tempboxa}%
    \repeat
  }
  \def\removehboxes{\setbox\FN@tempboxa=\lastbox
    \ifhbox
      \FN@tempboxa{\removehboxes}%
      \unhbox\FN@tempboxa
    \fi
  }
\fi
%    \end{macrocode}
% \end{macro}
% \end{macro}
%
%
% \subsection{The other footnote commands}\label{sec:perpage-code}
%
%    \begin{macrocode}
\ifFN@perpage
  \RequirePackage{perpage}
  \MakePerPage{footnote}
\fi
%    \end{macrocode}
%
% Finally, if we're not doing paragraph footnotes, we redefine
% \cs{@makefntext} to take account of the value of
% \cs{footnotemargin}, to impose \cs{footnotelayout}, and to make the
% footnote body text hang, if appropriate.
%    \begin{macrocode}
\ifFN@para
\else
%    \end{macrocode}
%
% hanging footnote version:
%    \begin{macrocode}
  \long\def\@makefntext#1{%
    \ifFN@hangfoot
      \bgroup
%    \end{macrocode}
%
% get the marker so we can measure it:
%    \begin{macrocode}
      \setbox\@tempboxa\hbox{%
        \ifdim\footnotemargin>0pt
          \hb@xt@\footnotemargin{\textfootmark{\@thefnmark}\hss}%
        \else
          \textfootmark{\@thefnmark}
        \fi
      }%
%    \end{macrocode}
%
% use the width of the box to set up hanging (potentially for more
% than one paragraph); note that the hanging \cs{parskip} and
% \cs{parindent} are set \emph{after} we've executed \cs{leavevmode}(!)
%    \begin{macrocode}
      \leftmargin\wd\@tempboxa
      \rightmargin\z@
      \linewidth \columnwidth
      \advance \linewidth -\leftmargin
      \parshape \@ne \leftmargin \linewidth
      \footnotesize
%    \end{macrocode}
%
% stop the \cs{parshape} being overwritten:
%    \begin{macrocode}
      \@setpar{{\@@par}}%
%    \end{macrocode}
%
% and finally put the marker in its chosen place:
%    \begin{macrocode}
      \leavevmode
      \llap{\box\@tempboxa}%
      \parskip\hangfootparskip\relax
      \parindent\hangfootparindent\relax
    \else
%    \end{macrocode}
%
% ordinary (non-hanging) footnote version:
%    \begin{macrocode}
      \parindent1em
      \noindent
      \ifdim\footnotemargin>\z@
        \hb@xt@ \footnotemargin{\hss\textfootmark{\@thefnmark}}%
      \else
        \ifdim\footnotemargin=\z@
          \llap{\textfootmark{\@thefnmark}}%
        \else
          \llap{\hb@xt@ -\footnotemargin{\textfootmark{\@thefnmark}\hss}}%
        \fi
      \fi
    \fi
    \footnotelayout#1%
%    \end{macrocode}
%
% if we're hanging, close the hang group
%    \begin{macrocode}
    \ifFN@hangfoot
      \par\egroup
    \fi
  }
\fi
%    \end{macrocode}
%
% \section{Remaining requirements}
%
% We have to insert the code that executes the \texttt{stable} and
% \texttt{multiple} options.  Since \texttt{stable} may suppress the
% setting of a footnote altogether, we put the \texttt{multiple}
% option first, as otherwise we might get isolated superscripted
% commas that separate footnotes that have otherwise been suppressed.
%
% \subsection{The code that executes the \texttt{multiple} option}
%
% \begin{macro}{\multiplefootnotemarker}
% \begin{macro}{\multfootsep}
% \begin{macro}{\@footnotemark}
% \begin{macro}{\FN@mf@prepare}
% \begin{macro}{\FN@mf@check}
% This (revised) code derives from a suggestion by Alexander Rozhenko
% (the author of the \textit{manyfoot} package): the intention is that
% \textit{footmisx} and \textit{manyfoot} should be able to
% `interwork', in the sense that each would recognise the other's
% footnote marks and behave appropriately.  The trick is that
% both \cs{footnote}  and \cs{footnotemark} insert a marker (a
% cancelling pair of kerns of \cs{multiplefootnotemarker} (of opposite
% signs), which is detected in following \cs{footnote} or
% \cs{footnotemark} commands.  Note we have to take special
% precautions to ensure that the kerns are the last things added to
% the horizontal list by the commands.
%    \begin{macrocode}
\ifFN@multiplefootnote
  \providecommand*{\multiplefootnotemarker}{3sp}
  \providecommand*{\multfootsep}{,}
  \def\FN@mf@prepare{%
    \kern-\multiplefootnotemarker
    \kern\multiplefootnotemarker\relax
  }
  \def\FN@mf@check{%
    \ifdim\lastkern=\multiplefootnotemarker\relax
      \edef\@x@sf{\the\spacefactor}%
      \unkern
      \textsuperscript{\multfootsep}%
      \spacefactor\@x@sf\relax
    \fi
  }
%    \end{macrocode}
%
% If we're not doing multiple, just create an empty \cs{FN@mf@prepare}
%    \begin{macrocode}
\else
  \let\FN@mf@check\relax
  \let\FN@mf@prepare\relax
\fi
%    \end{macrocode}
% \end{macro}
% \end{macro}
% \end{macro}
% \end{macro}
% \end{macro}
%
% \subsection{The code that executes the \texttt{stable} option}
%
% \begin{macro}{\ifFN@stablefootnote}
% \begin{macro}{\FN@sf@footmisx@footnote}
% The basic idea is to use the `original' code of \cs{footnote} (which
% this package may have hacked around something chronic) only if we're
% in typesetting mode (as determined by the state of the \cs{protect}
% command.  Otherwise, the command becomes an elaborate multistage
% `gobble'.
%    \begin{macrocode}
\ifFN@stablefootnote
\let\FN@sf@footmisx@footnote\footnote
\def\footnote{\ifx\protect\@typeset@protect
    \expandafter\FN@sf@footmisx@footnote
  \else
    \expandafter\FN@sf@gobble@opt
  \fi
}
%    \end{macrocode}
% \end{macro}
% \end{macro}
%
% \begin{macro}{\FN@sf@gobble@opt}
% \begin{macro}{\FN@sf@gobble@twobracket}
% Define \cs{FN@sf@gobble@opt} as a robust command that gobbles either
% an optional and a mandatory argument, or just a mandatory one.
%    \begin{macrocode}
\edef\FN@sf@gobble@opt{\noexpand\protect
  \expandafter\noexpand\csname FN@sf@gobble@opt \endcsname}
\expandafter\def\csname FN@sf@gobble@opt \endcsname{%
  \@ifnextchar[%]
    \FN@sf@gobble@twobracket
    \@gobble
}
\def\FN@sf@gobble@twobracket[#1]#2{}
%    \end{macrocode}
% \end{macro}
% \end{macro}
%
% \begin{macro}{\FN@sf@footmisx@footnotemark}
% \begin{macro}{\FN@sf@gobble@optonly}
% \begin{macro}{\FN@sf@gobble@bracket}
% Now the same for \cs{footnotemark}
%    \begin{macrocode}
\let\FN@sf@footmisx@footnotemark\footnotemark
\def\footnotemark{\ifx\protect\@typeset@protect
    \expandafter\FN@sf@footmisx@footnotemark
  \else
    \expandafter\FN@sf@gobble@optonly
  \fi
}
\edef\FN@sf@gobble@optonly{\noexpand\protect
  \expandafter\noexpand\csname FN@sf@gobble@optonly \endcsname}
\expandafter\def\csname FN@sf@gobble@optonly \endcsname{%
  \@ifnextchar[%]
    \FN@sf@gobble@bracket
    {}%
}
\def\FN@sf@gobble@bracket[#1]{}
\fi
%    \end{macrocode}
% \end{macro}
% \end{macro}
% \end{macro}
%
% \begin{macro}{\setfnsymbol}
% \begin{macro}{\FN@fnsymbol@lamport}
% \section{Symbol option variants}
%
% Lamport's choice of symbols for \cs{fnsymbol} wasn't entirely
% ``traditional'', so we (now) provide alternatives.  The
% \cs{setfnsymbol} command offers a small number of choices, and the
% user may define more still, using the \cs{DefineFNsymbols} or
% \cs{DefineFNsymbolsTM} commands, defined below.
%    \begin{macrocode}
\newcommand\setfnsymbol[1]{%
  \@bsphack
  \@ifundefined{FN@fnsymbol@#1}%
  {%
    \PackageError{footmisx}{Symbol style "#1" not known}%
    \@eha
  }{%
    \expandafter\let\expandafter\@fnsymbol\csname
                        FN@fnsymbol@#1\endcsname
  }%
  \@esphack
}
%    \end{macrocode}
%
% The default selection is Lamport's original, as represented in
% current \LaTeX{}~--- we preserve it in case we need to ``get back''
% to it.
%    \begin{macrocode}
\let\FN@fnsymbol@lamport\@fnsymbol
%</driver|package>
%    \end{macrocode}
% \end{macro}
% \end{macro}
%
% \begin{macro}{\if@tempswb}
% \begin{macro}{\@tempswbfalse}
% \begin{macro}{\@tempswbtrue}
% We need another temp conditional
%    \begin{macrocode}
\newif\if@tempswb
%    \end{macrocode}
% \end{macro}
% \end{macro}
% \end{macro}
%
% \begin{macro}{\DefineFNsymbols}
% \begin{macro}{\@DefineFNsymbols}
% \begin{macro}{\@DefineFNsymbols@}
% \begin{macro}{\FN@build@symboldef}
% The macro \cs{DefineFNsymbols} allows the user to define a set of
% footnote symbols, to be used with the \cs{setfnsymbol} command.
% Syntax:\par\noindent
% \cs{DefineFNsymbols}|[*]|\marg{set name}\oarg{style}\marg{symbol list}
%
% If the optional asterisk is present, the set defined will produce an
% error if the symbol number is too large; otherwise it will quietly
% change to numbering in place of symbol use (a warning is produced at
% the end of the document). The set name is the future argument of
% \cs{setfnsymbol}).  The style (default \texttt{text}) gives the style
% the symbols are typeset (this is the \emph{correct} method, but
% unfortunately not  all symbols, even for Lamport's original set for
% \LaTeX{} \cs{fnsymbol} may be expressed this way in a sufficiently
% old \LaTeX{} distribution).  The symbol list is a set of objects to
% be used when the set is selected.
%
% Example of use:\par\noindent define a direct replacement for
% Lamport's original \cs{fnsymbol} command ---
%\begin{verbatim}
%\DefineFNsymbols*{lamport}[math]{*\dagger\ddagger\mathsection
%  \mathparagraph\|{**}{\dagger\dagger}{\ddagger\ddagger}%
%}
%\end{verbatim}
% Note that doubled-up (and worse\,---\,see below) symbols need braces
% around them.
%    \begin{macrocode}
\newcommand{\DefineFNsymbols}{%
        \@ifstar{\@tempswbtrue\@DefineFNsymbols}%
                {\@tempswbfalse\@DefineFNsymbols}%
}
\newcommand{\@DefineFNsymbols}[1]{%
  \@ifnextchar[% ]
    {\@DefineFNsymbols@{#1}}{\@DefineFNsymbols@{#1}[text]}%
}
\def\@DefineFNsymbols@#1[#2]#3{%
  \expandafter\ifx\csname FN@fnsymbol@#1\endcsname\relax
    \PackageInfo{footmisx}{Declaring symbol style #1}%
  \else
    \PackageWarning{footmisx}{Redeclaring symbol style #1}%
  \fi
  \toks@{}%
  \def\@tempb{\end}%
  \FN@build@symboldef#3\end
  \def\@tempc{math}%
  \def\@tempd{#2}%
  \expandafter\xdef\csname FN@fnsymbol@#1\endcsname##1{%
    \ifx\@tempc\@tempd
      \noexpand\ensuremath
    \else
      \noexpand\nfss@text
    \fi
    {%
      \noexpand\ifcase##1%
      \the\toks@
      \noexpand\else
      \if@tempswb
        \noexpand\@ctrerr
      \else
        \noexpand\@arabic##1\noexpand\FN@orange##1%
      \fi
      \noexpand\fi
    }%
  }%
}
\def\FN@build@symboldef#1{%
  \def\@tempa{#1}%
  \ifx\@tempa\@tempb
  \else
    \toks@\expandafter{\the\toks@\or#1}%
    \expandafter\FN@build@symboldef
  \fi
}
%    \end{macrocode}
% \end{macro}
% \end{macro}
% \end{macro}
% \end{macro}
%
% \begin{macro}{\DefineFNsymbolsTM}
% \begin{macro}{\@DefineFNsymbolsTM}
% \begin{macro}{\FN@build@symboldefTM}
%
% Now do the same job for the ``modern'' way of having both text and
% maths variants of everything.
%    \begin{macrocode}
\newcommand{\DefineFNsymbolsTM}{%
        \@ifstar{\@tempswbtrue\@DefineFNsymbolsTM}%
                {\@tempswbfalse\@DefineFNsymbolsTM}}%
\newcommand{\@DefineFNsymbolsTM}[2]{%
  \expandafter\ifx\csname FN@fnsymbol@#1\endcsname\relax
    \PackageInfo{footmisx}{Declaring symbol style #1}%
  \else
    \PackageWarning{footmisx}{Redeclaring symbol style #1}%
  \fi
  \toks@{}%
  \def\@tempb{\end}%
  \FN@build@symboldefTM#2\end\@null
  \expandafter\xdef\csname FN@fnsymbol@#1\endcsname##1{%
    \noexpand\ifcase##1%
      \the\toks@
    \noexpand\else
      \if@tempswb
        \noexpand\@ctrerr
      \else
        \noexpand\@arabic##1\noexpand\FN@orange##1%
      \fi
    \noexpand\fi
  }%
}%
%    \end{macrocode}
% Note that this version has two variants of every definition, so
% needs two stopper codes above.
%    \begin{macrocode}
\def\FN@build@symboldefTM#1#2{%
  \def\@tempa{#1}%
  \ifx\@tempa\@tempb
  \else
    \toks@\expandafter{\the\toks@\or\TextOrMath{#1}{#2}}%
    \expandafter\FN@build@symboldefTM
  \fi
}
%    \end{macrocode}
% \end{macro}
% \end{macro}
% \end{macro}
%
% \begin{macro}{\TextOrMath}
% This is a stripped down (e-\TeX{} only) version of what appears in
% fixltx2e.  If the command's already defined, we assume it's that
% version.
%    \begin{macrocode}
\@ifundefined{TextOrMath}{%
  \@ifundefined{eTeXversion}{%
    \PackageError{footmisx}{Can't define commands for footnote symbol}%
                           {Use e-LaTeX, or load package fixltx2e before
                             footmisx}%
  }{%
    \protected\expandafter\def\csname TextOrMath\space\endcsname{%
      \ifmmode \expandafter\@secondoftwo
      \else    \expandafter\@firstoftwo   \fi
    }
    \edef\TextOrMath#1#2{%
      \expandafter\noexpand\csname TextOrMath\space\endcsname
        {#1}{#2}%
    }%
  }%
}{}
%    \end{macrocode}
% \end{macro}
%
% \begin{macro}{\FN@orange}
% \begin{macro}{\@fnsymbol@orange}
% \begin{macro}{\@diagnose@fnsymbol@orange}
% Macros to deal with footnote symbols going out of range (when
% they're allowed to\,--\,e.g., in the \texttt{symbol*} option).
%    \begin{macrocode}
\def\FN@orange#1{%
  \@bsphack
  \PackageInfo{footmisx}{Footnote number \number#1 out of range}%
  \protect\@fnsymbol@orange
  \@esphack
}
\global\let\@diagnose@fnsymbol@orange\relax
\AtEndDocument{\@diagnose@fnsymbol@orange}
\def\@fnsymbol@orange{%
  \gdef\@diagnose@fnsymbol@orange{%
    \PackageWarningNoLine{footmisx}{Some footnote number(s)
      were out of range
      \MessageBreak
      see log for details%
    }%
  }%
}
%    \end{macrocode}
% \end{macro}
% \end{macro}
% \end{macro}
%
% \begin{macro}{\textbardbl}
% This is defined in recent \LaTeX{} releases, but not in (for
% example) that distributed with the last release of te\TeX{}.  Since
% it's needed in some symbol set definitions (including Lamport's) we
% define it here.
%    \begin{macrocode}
\@ifundefined{textbardbl}{%
  \DeclareTextSymbol{\textbardbl}{OMS}{107}%
  \DeclareTextSymbolDefault{\textbardbl}{TS1}}{}%
%    \end{macrocode}
% (This definition comes from the \LaTeX{} sources.)
% \end{macro}
%
% \begin{macro}{\FN@fnsymbol@bringhurst}
% \begin{macro}{\FN@fnsymbol@chicago}
% \begin{macro}{\FN@fnsymbol@wiley}
% \begin{macro}{\FN@fnsymbol@lamport-robust}
% \begin{macro}{\FN@fnsymbol@lamport}
% These macros provide replacement orderings (and symbol sets) for
% footnote symbols, plus a robust version of the original Lamport set,
% and an extended version of Lamport's original
%    \begin{macrocode}
\DefineFNsymbolsTM*{bringhurst}{%
  \textasteriskcentered *
  \textdagger    \dagger
  \textdaggerdbl \ddagger
  \textsection   \mathsection
  \textbardbl    \|%
  \textparagraph \mathparagraph
}%
\DefineFNsymbolsTM*{chicago}{%
  \textasteriskcentered *
  \textdagger    \dagger
  \textdaggerdbl \ddagger
  \textsection   \mathsection
  \textbardbl    \|%
  \#\#%
}%
\DefineFNsymbolsTM*{wiley}{
  \textasteriskcentered *
  {\textasteriskcentered\textasteriskcentered}{**}%
  \textdagger    \dagger
  \textdaggerdbl \ddagger
  \textsection   \mathsection
  \textparagraph \mathparagraph
  \textbardbl    \|%
}%
\DefineFNsymbolsTM{lamport-robust}{
  \textasteriskcentered *
  \textdagger    \dagger
  \textdaggerdbl \ddagger
  \textsection   \mathsection
  \textparagraph \mathparagraph
  \textbardbl    \|%
  {\textasteriskcentered\textasteriskcentered}{**}%
  {\textdagger\textdagger}{\dagger\dagger}%
  {\textdaggerdbl\textdaggerdbl}{\ddagger\ddagger}%
}
\DefineFNsymbolsTM*{lamport*}{%
  \textasteriskcentered *
  \textdagger    \dagger
  \textdaggerdbl \ddagger
  \textsection   \mathsection
  \textparagraph \mathparagraph
  \textbardbl    \|%
  {\textasteriskcentered\textasteriskcentered}{**}%
  {\textdagger\textdagger}{\dagger\dagger}%
  {\textdaggerdbl\textdaggerdbl}{\ddagger\ddagger}%
  {\textsection\textsection}{\mathsection\mathsection}%
  {\textparagraph\textparagraph}{\mathparagraph\mathparagraph}%
  {\textasteriskcentered\textasteriskcentered\textasteriskcentered}{***}%
  {\textdagger\textdagger\textdagger}{\dagger\dagger\dagger}%
  {\textdaggerdbl\textdaggerdbl\textdaggerdbl}{\ddagger\ddagger\ddagger}%
  {\textsection\textsection\textsection}%%
    {\mathsection\mathsection\mathsection}%
  {\textparagraph\textparagraph\textparagraph}%%
    {\mathparagraph\mathparagraph\mathparagraph}%
}
\setfnsymbol{lamport*}
\DefineFNsymbolsTM{lamport*-robust}{%
  \textasteriskcentered *
  \textdagger    \dagger
  \textdaggerdbl \ddagger
  \textsection   \mathsection
  \textparagraph \mathparagraph
  \textbardbl    \|%
  {\textasteriskcentered\textasteriskcentered}{**}%
  {\textdagger\textdagger}{\dagger\dagger}%
  {\textdaggerdbl\textdaggerdbl}{\ddagger\ddagger}%
  {\textsection\textsection}{\mathsection\mathsection}%
  {\textparagraph\textparagraph}{\mathparagraph\mathparagraph}%
  {\textasteriskcentered\textasteriskcentered\textasteriskcentered}{***}%
  {\textdagger\textdagger\textdagger}{\dagger\dagger\dagger}%
  {\textdaggerdbl\textdaggerdbl\textdaggerdbl}{\ddagger\ddagger\ddagger}%
  {\textsection\textsection\textsection}%%
    {\mathsection\mathsection\mathsection}%
  {\textparagraph\textparagraph\textparagraph}%%
    {\mathparagraph\mathparagraph\mathparagraph}%
}
%    \end{macrocode}
% \end{macro}
% \end{macro}
% \end{macro}
% \end{macro}
% \end{macro}
%
%
% \section{Core function}
%
% \subsection{Tools}
%
% Here we define some tools functions.
%
% \subsubsection{\cs{@@_noHref:n}}
%
% First we define a pointer
% to \tn{ref*} if \texttt{hyperref} is loaded or \tn{ref} if not
% loaded.
% \begin{function}{\@@_noHref:n}
% \begin{syntax}
%   \cs{\@@_noHref:n} \Arg{label}
% \end{syntax}
%    \begin{macrocode}
\ExplSyntaxOn
\cs_new_nopar:Nn \@@_noHref:n {\ref{#1}}
%    \end{macrocode}
% \end{function}
% We run this code now (update) and eventually at end of hyperref file
% \begin{function}{\@@_define_@@_noHref:}
% \begin{syntax}
%   \cs{@@_define_@@_noHref:}
% \end{syntax}
% Define \cs{@@_noHref:n} to something sensible.
%    \begin{macrocode}
\tl_const:Nn \@@_define_@@_noHref: {
  \@ifpackageloaded{hyperref}{
    \cs_gset:Nn \@@_noHref:n {\ref*{#1}}
  }
  {
    \cs_gset:Nn \@@_noHref:n {\ref{#1}}
  }
}
%    \end{macrocode}
% \end{function}
% Now redefine this function at end of hyperref thus allowing
% independant order of loading between \textsf{hyperref} and
% \textsf{footmisx}
%    \begin{macrocode}
\@@_AtEndOfPackageFile_and_now:NN {hyperref}{
  \@@_define_@@_noHref: {}
}
%    \end{macrocode}
%
% \subsection{\cs{@@_makefnmarklink}}
%
% This function create a footnote mark including an a link if hyperref is loaded.
% \begin{function}{\@@_@makefnmarklink:n}
%   \begin{syntax}
%   \cs{@@_@makefnmarklink:n} \Arg{marktoset}
%   \end{syntax}
%    Typeset a footnote mark including link. Will be overriden by a hook later.
%    \begin{macrocode}
\cs_new_nopar:Nn \@@_@makefnmarklink:n {\textfootmark{#1}}
%    \end{macrocode}
% \end{function}
% We define the hyperref hook run when hyperref is not loaded
% \begin{function}{\@@_noref_@makefnmarklink}
%   \begin{syntax}
%     \cs{@@_noref_@makefnmarklink} \Arg{marktoset}
%   \end{syntax}
%   Set a mark anchor in case of hyperref
%    \begin{macrocode}
\cs_new_eq:NN \@@_noref_@makefnmarklink:n \@@_@makefnmarklink:n
%    \end{macrocode}
% \end{function}
% We define the hyperref hook
% \begin{function}{\@@_hyperref_@makefnmarklink:n}
%   \begin{syntax}
%     \cs{@@_hyperref_@makefnmarklink:n} \Arg{marktoset}
%   \end{syntax}
%   Set a mark anchor in case of hyperref
%    \begin{macrocode}
\cs_new_nopar:Nn \@@_hyperref_@makefnmarklink:n {
    \@@_noref_@makefnmarklink:n {
    \stepcounter{Hfootnote}%
    \global\let\Hy@saved@currentHref\@currentHref
    \hyper@makecurrent{Hfootnote}%
    \global\let\Hy@footnote@currentHref\@currentHref
    \global\let\@currentHref\Hy@saved@currentHref
    \hyper@linkstart{link}{\Hy@footnote@currentHref}%
    #1
    \hyper@linkend
    }
}
%    \end{macrocode}
% \end{function}
% We run this hook now, and at end of hyperref package
% \begin{function}{\@@_define_@@_@makefnmarklink:}
% \begin{syntax}
%   \cs{@@_define_@@_@makefnmarklink:}
% \end{syntax}
% Define \cs{@@_@makefnmarklink} to something sensible.
%    \begin{macrocode}
\tl_const:Nn \@@_define_@@_@makefnmarklink: {
  \@ifpackageloaded{hyperref}{
    \ifHy@hyperfootnotes
       \cs_gset_eq:NN \@@_@makefnmarklink:n \@@_hyperref_@makefnmarklink:n
    \else
       \cs_gset_eq:NN \@@_@makefnmarklink:n \@@_noref_@makefnmarklink:n
    \fi
  }
  {
    \cs_gset_eq:NN \@@_@makefnmarklink:n \@@_noref_@makefnmarklink:n
  }
}
%    \end{macrocode}
% \end{function}
% Now run the hook
%    \begin{macrocode}
\@@_AtEndOfPackageFile_and_now:NN {hyperref} {
  \@@_define_@@_@makefnmarklink: {}
}
%    \end{macrocode}
% Define now a \textsf{xparse} helper:
% \begin{function}{\@@_@makefnmarklink_stared:Nn}
%  \begin{syntax}
%    \cs{@@_@makefnmarklink_stared:Nn} \marg{value or \texttt{-novalue}-} \marg{mark}
%  \end{syntax}
% If first argument is defined call no hyperlink version, else call hyperlink
% version.
%    \begin{macrocode}
\cs_new_nopar:Nn \@@_@makefnmarklink_stared:Nn {
  \IfBooleanTF{#1}
    {\@@_noref_@makefnmarklink:n{#2}}
    {\@@_@makefnmarklink:n{#2}}
}
%    \end{macrocode}
% \end{function}
% Now define a the xparse handler
% \begin{function}{\@@_makefnmarklink}
% \begin{syntax}
% \cs{\@@_makefnmarklink}|[*]|\oarg{mark=\tn{@thefnmark}}
% \end{syntax}
%    \begin{macrocode}
\DeclareDocumentCommand{\@@_makefnmarklink}
  {sO{\@thefnmark}}
  {\@@_@makefnmarklink_stared:Nn{#1}{#2}
}
%    \end{macrocode}
% \end{function}
 
% \subsection{\tn{@makefnmark}}
%
% \tn{makefnmark} is unfixable, it is used both for a link and a target. We try our best
% and follow hyperref. By default \tn{@makefnmark} is a link (using the same approach 
% than \textsl{hyperref}).
%
% We begin by saving the original \tn{@makefnmark}.
% \begin{function}{\@@_orig_@makefnmark}
%  \begin{syntax}
%    \cs{@@_orig_@makefnmark}
%  \end{syntax}
%    \begin{macrocode}
\cs_set_eq:NN \@@_orig_@makefnmark \@makefnmark
%    \end{macrocode}
% \end{function}
% Now we redefine \tn{@makefnmark} in compatible fashion
% \begin{function}{\@makefnmark}
% \begin{syntax}
% \tn{\@@_makefnmarklink}|[*]|\oarg{mark=\tn{@thefnmark}}
% \end{syntax}
%    \begin{macrocode}
\cs_set_eq:NN \@makefnmark \@@_makefnmarklink
%    \end{macrocode}
% \end{function}
%
% \subsection{\tn{@footnotemark}}
%
% Change \tn{@footnotemark} to allow a star version (no hyperref) and an optional ref.
%
%    \begin{macrocode}
\ExplSyntaxOn
\cs_new_nopar:Nn \@@_@footnotemark_stared:Nn {
  \leavevmode
  \ifhmode
    \edef\@x@sf{\the\spacefactor}
    \FN@mf@check
    \nobreak
  \fi
  \@@_@makefnmarklink_stared:Nn{#1}{#2}
  \FN@mf@prepare
  \ifhmode\spacefactor\@x@sf\fi
  \relax
}
\DeclareDocumentCommand{\@@_@footnotemark}{sO{\@thefnmark}}{
   \@@_@footnotemark_stared:Nn{#1}{#2}
}
\tl_const:Nn \@@_define_@footnotemark: {
   \cs_gset_eq:NN \@footnotemark \@@_@footnotemark
}
\@@_AtEndOfPackageFile_and_now:NN {hyperref} {
  \@@_define_@footnotemark: {}
}
\ExplSyntaxOff
%    \end{macrocode}
%
% \subsection{\tn{footnotemark}}

%
% \subsection{\tn{@footnotetext}}
%
% \subsubsection{\textsl{setspace} package}
%
% \begin{macro}{\ifFN@baselinestretch}
% \begin{macro}{\FN@singlespace}
% Whatever we do, we are going to patch \cs@{footnotetext}; so first
% of all, we'll check it's not been hacked by anyone other than
% \texttt{setspace.sty} (while we're at it we also record whether
% \texttt{setspace} is loaded).
% so we do this here:
%    \begin{macrocode}
\ExplSyntaxOn
\newif\ifFN@setspace
\@@_AtEndOfPackageFile_and_now:NN {setspace} {
\@ifpackageloaded{setspace}{%
  \FN@setspacetrue
  \@ifclassloaded{memoir}{%
%    \end{macrocode}
% we're seeing \textsf{memoir}'s emulation of \textsf{setspace}
%    \begin{macrocode}
    \let\FN@baselinestretch\m@m@singlespace
  }{%
%    \end{macrocode}
% we're seeing \textsf{setspace} in its own right
%    \begin{macrocode}
    \let\FN@baselinestretch\setspace@singlespace
  }%
}{%
  \FN@setspacefalse
}
}
\ExplSyntaxOff
%    \end{macrocode}
% \end{macro}
% \end{macro}
%
% \subsubsection{Command}
%
% There's substantial patching to be done if we're doing paragraph
% footnotes:
%    \begin{macrocode}
\ifFN@sidefn
    \newcommand\@footmisxnotetext[1]{%
      \marginpar{%
%    \end{macrocode}
% insert compatibility code with |setspace.sty| if necessary
%    \begin{macrocode}
        \ifFN@setspace
          \let\baselinestretch\FN@baselinestretch
        \fi
        \reset@font\footnotesize
        \protected@edef\@currentlabel{%
          \csname p@footnote\endcsname\@thefnmark
        }%
        \color@begingroup
          \@makefntext{%
            \ignorespaces#1%
          }%
        \color@endgroup
      }%
      \FN@mf@prepare
    }%
\else
\ifFN@para
  \newcommand\@footmisxnotetext[1]{%
    \insert\footins{%
%    \end{macrocode}
% insert compatibility code with |setspace.sty| if necessary
%    \begin{macrocode}
      \ifFN@setspace
        \let\baselinestretch\FN@baselinestretch
      \fi
      \reset@font\footnotesize
      \interlinepenalty\interfootnotelinepenalty
      \splittopskip\footnotesep
      \splitmaxdepth \dp\strutbox
      \floatingpenalty\@MM
      \hsize\columnwidth
      \@parboxrestore
      \protected@edef\@currentlabel{\csname p@footnote\endcsname\@thefnmark}%
      \color@begingroup
%    \end{macrocode}
%
% We set the paragraph in an \cs{hbox} and apply the fudge factor
% here:
%
%    \begin{macrocode}
        \setbox\FN@tempboxa=\hbox{%
%    \end{macrocode}
%
% This needs a parameter; the rule should be moved to the beginning of
% the footnote paragraph, but the \cs{ignorespaces} should be left
% here.
%
%    \begin{macrocode}
          \@makefntext{\ignorespaces#1\strut
%    \end{macrocode}
%
% We insert a penalty here to help line breaking in the
% footnote paragraph; the value is taken from the \TeX{}book.
%
%    \begin{macrocode}
            \penalty-10\relax
            \hskip\footglue
          }% end of \@makefntext parameter
        }% end of \hbox
        \dp\FN@tempboxa=0pt
        \ht\FN@tempboxa=\dimexpr\wd\FN@tempboxa *
                          \footnotebaselineskip / \columnwidth\relax
        \box\FN@tempboxa
      \color@endgroup
    }%
    \FN@mf@prepare
  }
%    \end{macrocode}
%
% If we're not doing paragraph footnotes, we now simply tag a
% \cs{FN@mf@prepare} command on the end of the definition; of course,
% there are different definitions according as whether we're using
% |side| footnotes\dots
%    \begin{macrocode}
  \else
    \newcommand\@footmisxnotetext[1]{%
      \insert\footins{%
%    \end{macrocode}
% insert compatibility code with \textsf{setspace} if necessary
%    \begin{macrocode}
        \ifFN@setspace
          \let\baselinestretch\FN@baselinestretch
        \fi
        \reset@font\footnotesize
        \interlinepenalty\interfootnotelinepenalty
        \splittopskip\footnotesep
        \splitmaxdepth \dp\strutbox
        \floatingpenalty\@MM
        \hsize\columnwidth
        \@parboxrestore
        \protected@edef\@currentlabel{%
          \csname p@footnote\endcsname\@thefnmark
        }%
        \color@begingroup
          \@makefntext{%
            \rule\z@\footnotesep
            \ignorespaces#1\@finalstrut\strutbox
          }%
        \color@endgroup
      }%
      \FN@mf@prepare
    }%
  \fi
\fi
%    \end{macrocode}
% Now define \cs{@footnotetext} to be \cs{@footmisxnotetext}. 
% If setspace was loaded before \textsf{hyperref} we revert the change
% by using our code as \textsf{hypperef} hook.
%    \begin{macrocode}
\ExplSyntaxOn
\@@_AtEndOfPackageFile_and_now:NN {hyperref} {
  \@ifpackageloaded{hyperref}
  {
    \cs_gset_eq:cc {H @ @ footnotetext} {@footmisxnotetext}
  }{
    \cs_gset_eq:cc {@footnotetext} {@footmisxnotetext}
  }
}
\ExplSyntaxOff
%    \end{macrocode}
% Disable \textsf{setspace} hook.
%    \begin{macrocode}
\ExplSyntaxOn
\AtBeginOfPackageFile{setspace}{
  \cs_gset_eq:cc {@@_saved_setsave_@footnotetext} {@footnotetext}
  \cs_gset_eq:cc {@@_saved_setsave_@mpfootnotetext} {@mpfootnotetext}
}%
\AtEndOfPackageFile{setspace}{
  \cs_gset_eq:cc {@footnotetext} {@@_saved_setsave_@footnotetext}
  \cs_gset_eq:cc {@mpfootnotetext} {@@_saved_setsave_@mpfootnotetext}
}
\ExplSyntaxOff
%    \begin{macrocode}
%    \end{macrocode}
%
% \section{Other miscellaneous commands}
%
% \subsection{Footnote mark style}
%
% \begin{macro}{\textfootmark}
% Syntax \cs{textfootmark}\marg{text}
%
% Typographically set the text in the same style than a raw
% footnotemark (it is in fact a generic version of \tn{@makefnmark})
%    \begin{macrocode}
\ExplSyntaxOn
\NewDocumentCommand{\textfootmark}{m}
    {\hbox{\@textsuperscript{\normalfont#1}}}
\ExplSyntaxOff
%    \end{macrocode}
% \end{macro}
%
% \subsection{Footnote references}
%
%
% Syntax: \cs{footref}\marg{label-name}
%
% One often wishes to refer to a footnote; in some circumstances,
% \cs{footnotemark} just isn't good enough (for example, inside a
% |minipage|, when \cs{footnotemark} creates a reference to footnotes
% outside the minipage).
%
% \cs{footref} addresses this problem by making a label reference that
% actually looks like a \cs{footnotemark}.  (The command is available
% in the |memoir| class, and we therefore \cs{providecommand} it
% rather than defining it ``outright''.)
%    \begin{macrocode}
\ExplSyntaxOn
\cs_new_nopar:Nn \@@_footref_stared:Nn {
  \textfootmark{
    \IfBooleanTF{#1}
      {\@@_noHref:n{#2}}
      {\ref{#2}}
  }
}
\DeclareDocumentCommand {\@@_footref} {s m}
{
    \@@_footref_stared:Nn{#1}{#2}
}
\ExplSyntaxOff
%    \end{macrocode}
% We run hook now and at end of hyperref
%    \begin{macrocode}
\ExplSyntaxOn
\@@_AtEndOfPackageFile_and_now:NN {hyperref} {\cs_gset_eq:NN \footref \@@_footref}
\ExplSyntaxOff
%    \end{macrocode}
%
% \subsection{Minipage \cs{footnotemark}s}
%
% \begin{macro}{\mpfootnotemark}
% Syntax: \cs{mpfootnotemark}\oarg{number}
%
% Here we define \cs{mpfootnotemark}, which has the same syntax as
% \cs{footnotemark}, and which applies the semantics of
% \cs{footnotemark} to the minipage footnote series.
%    \begin{macrocode}
\ExplSyntaxOn
\DeclareDocumentCommand\mpfootnotemark{o}
{
  \IfValueTF {#1}
  {
    \group_begin:
    \cs_set:cpn {c@\@mpfn} {#1}
    \unrestored@protected@xdef\@thefnmark{\thempfn}
    \group_end:
  }
  {
    \stepcounter\@mpfn
    \protected@xdef\@thefnmark{\thempfn}
  }
  \@footnotemark
}
\ExplSyntaxOff
%    \end{macrocode}
% \end{macro}
%
%    \begin{macrocode}
\endinput
%</package>
%    \end{macrocode}
%
% \Finale
%
%
%% \CharacterTable
%%  {Upper-case    \A\B\C\D\E\F\G\H\I\J\K\L\M\N\O\P\Q\R\S\T\U\V\W\X\Y\Z
%%   Lower-case    \a\b\c\d\e\f\g\h\i\j\k\l\m\n\o\p\q\r\s\t\u\v\w\x\y\z
%%   Digits        \0\1\2\3\4\5\6\7\8\9
%%   Exclamation   \!     Double quote  \"     Hash (number) \#
%%   Dollar        \$     Percent       \%     Ampersand     \&
%%   Acute accent  \'     Left paren    \(     Right paren   \)
%%   Asterisk      \*     Plus          \+     Comma         \,
%%   Minus         \-     Point         \.     Solidus       \/
%%   Colon         \:     Semicolon     \;     Less than     \<
%%   Equals        \=     Greater than  \>     Question mark \?
%%   Commercial at \@     Left bracket  \[     Backslash     \\
%%   Right bracket \]     Circumflex    \^     Underscore    \_
%%   Grave accent  \`     Left brace    \{     Vertical bar  \|
%%   Right brace   \}     Tilde         \~}
